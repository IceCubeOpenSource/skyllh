\documentclass{article}

\usepackage{xcolor}
\usepackage{amsmath}

\newcommand{\eq}[1]{(\ref{#1})}

\newcommand{\code}[1]{\texttt{#1}}
\newcommand{\class}[1]{\colorbox{blue!30}{\code{#1}}}

\newcommand{\ns}{n_{\mathrm{s}}}
\newcommand{\nsj}{n_{\mathrm{s}_j}}
\newcommand{\hatns}{\hat{n}_{\mathrm{s}}}
\newcommand{\ps}{\vec{p}_{\mathrm{s}}}
\newcommand{\psk}{\vec{p}_{\mathrm{s}_k}}
\newcommand{\hatps}{\vec{\hat{p}}_{\mathrm{s}}}
\newcommand{\xs}{\vec{x}_{\mathrm{s}}}
\newcommand{\xsk}{\vec{x}_{\mathrm{s}_k}}

\newcommand{\dPhisdE}{\frac{\mathrm{d}\Phi_{\mathrm{s}}}{\mathrm{d}E}}


\begin{document}

\section{The Hypotheses}

Before performing a statistical test, is is important to formulate the exact
hypotheses, which one wants to consider.

For the log-likelihood ratio tests that are considered here, a null-hypothesis
and an alternative hypothesis is required. As null-hypothesis a data set with
no signal events, i.e. with only background events, is usuallly considered.
As alternative hypothesis a data set including signal events from a given signal
(source) hypothesis is usuallly considered.

As general source hypothesis one could consider a source flux, $\Phi_{\mathrm{S}}$,
for instance a neutrino or gamma-ray particle flux from a source. Such a general flux
can be parameterized as
\begin{equation}
 \Phi_{\mathrm{S}}(\alpha,\delta,E,t|\xs, \ps),
 \label{eq:PhiS-general}
\end{equation}
which is a function of the celestial coordinates right-ascension, $\alpha$, and
declination, $\delta$, as well as the energy and time of the signal
particle, given the source to be at location $\xs$ with source
parameters $\ps$. At this stage $\xs$ doesn't have to be a single coordinate,
but could also describe an extended source, e.g. the galactic plane.

This flux is a differential flux, thus can be written as
\begin{equation}
 \frac{\mathrm{d}\Phi_{\mathrm{S}}(\alpha,\delta,E,t|\xs, \ps)}{\mathrm{d}A\mathrm{d}\Omega\mathrm{d}E\mathrm{d}t},
 \label{eq:differential-PhiS-general}
\end{equation}
where $A$ and $\Omega$ denotes area and solid-angle, respectively.

\subsection{Flux Models}
\label{sec:flux-models}

SkyLLH implements several flux model classes for $\Phi_{\mathrm{S}}$. The most
generic flux model class is \class{FluxModel}, which provides the abstract base
class for all flux models given by the differential flux
(\ref{eq:differential-PhiS-general}).

\subsection{Factorized Flux Models}
\label{sec:factorized-flux-models}

The traditional point-like source searches in IceCube use a factorized flux
model as signal hypothesis, where the spatial, energy, and time profiles of the
source factorize, i.e. are independent of each other, and a constant $\Phi_0$
provides the flux normalization. It should be noticed that such a
flux model is already an assumption on the source. Such a factorized flux model
is of the form
\begin{equation}
 \Phi_{\mathrm{S}}(\alpha,\delta,E,t|\xs,\ps) = \Phi_0 \Psi_{\mathrm{S}}(\alpha,\delta|\xs,\ps) \epsilon_{\mathrm{S}}(E|\ps) T_{\mathrm{S}}(t|\ps),
\end{equation}
where $\Phi_0$ is the flux normalization carrying the differential flux units,
and $\Psi_{\mathrm{S}}$, $\epsilon_{\mathrm{S}}$, and $T_{\mathrm{S}}$ are the
spatial, energy, and time profiles of the source, respectively.

In SkyLLH the Python class \class{FactorizedFluxModel}, derived from class
\class{FluxModel}, provides a base class for this mathematical class of flux models.

\subsection{Point-like Source Factorized Flux Models}
\label{sec:point-like-source-factorized-flux-models}

In IceCube the traditional searches for astro-physical neutrinos were searches
for neutrino emission from point-like objects in the sky, where a factorized
flux model has been used as source hypothesis. Thus, the spatial profile is
given by two delta functions:
\begin{equation}
 \Psi_{\mathrm{S}}(\alpha,\delta|\xs) = \delta(\alpha-\alpha_{\mathrm{s}})\delta(\delta-\delta_{\mathrm{s}}),
\end{equation}
where the source location $\xs$ is given by a single point in the sky in
equatorial coordinates: $\xs = (\alpha_{\mathrm{s}}, \delta_{\mathrm{s}})$.
Hence, the flux model can be formulated as:
\begin{equation}
 \Phi_{\mathrm{S}}(\alpha,\delta,E,t|\xs,\ps) = \Phi_0 \delta(\alpha-\alpha_{\mathrm{s}})\delta(\delta-\delta_{\mathrm{s}}) \epsilon_{\mathrm{S}}(E|\ps) T_{\mathrm{S}}(t|\ps).
\end{equation}

For such point-like source flux models SkyLLH provides the
\class{PointlikeSourceFFM} class, which is derived from the \class{FactorizedFluxModel}
class.

The emission energy-profile of the source flux might be assumed to be a power-law:
\begin{equation}
 \epsilon_{\mathrm{S}}(E|\ps) \equiv \epsilon_{\mathrm{S}}(E|\gamma) = \left(\frac{E}{100\mathrm{TeV}}\right)^{-\gamma},
\end{equation}
where $\gamma$ is the spectral index of the source.

As emission time-profile of the source, $T_{\mathrm{S}}(t|\ps)$, different functional
forms could be imagined.
A steady, i.e. time independent, source emission time profile would be
\begin{equation}
 T_{\mathrm{S}}(t|\ps) \equiv 1.
\end{equation}

In cases of a time variable source, $T_{\mathrm{S}}(t|\ps)$ could be a box
profile of length $T_{\mathrm{W}}$ with the box's middle time position, $T_{0}$,
\begin{equation}
 T_{\mathrm{S}}(t|T_0,T_{\mathrm{W}}) =
   \begin{cases}
     1 & \forall t \in \left[T_0 - T_{\mathrm{W}}/2; T_0 + T_{\mathrm{W}}/2 \right]\\
     0 & \mathrm{otherwise}
   \end{cases},
 \label{eq:Ts-box}
\end{equation}
or a Gaussian shaped time profile centered at $T_0$ with a time width of $\sigma_T$,
\begin{equation}
 T_{\mathrm{S}}(t|T_0,\sigma_T) \equiv \exp\left(-\frac{(t - T_0)^2}{2\sigma_T^2}\right).
 \label{eq:Ts-gauss-non-truncated}
\end{equation}
For efficiency reasons, the Gaussian shape source time profile could also be
truncated at a certain distance from $T_0$, \emph{e.g.} at $\pm\sigma_T$:
\begin{equation}
 T_{\mathrm{S}}(t|T_0,\sigma_T) \equiv
   \begin{cases}
     \exp\left(-\frac{(t - T_0)^2}{2\sigma_T^2}\right) & \left|t-T_0\right| \le \sigma_T\\
     0 & \mathrm{otherwise}
   \end{cases}.
 \label{eq:Ts-gauss-truncated}
\end{equation}


\section{The Likelihood Formalism}

This section describes the mathematical likelihood formalism used in SkyLLH.
First it introduces the log-likelihood approach, second the likelihood ratio
test and the used test statistic and then describes the used optimizations.

\subsection{The Log-Likelihood Approach}

SkyLLH implements the two-component likelihood approach with a likelihood
function $\mathcal{L}(n_{\mathrm{s}},\vec{p}_{\mathrm{s}}~|D)$ of the form
\begin{equation}
 \mathcal{L}(\ns,\ps~|D) = \prod_{i=1}^{N}\left[ \frac{\ns}{N} S_{i}(\ps) + (1 - \frac{\ns}{N}) B_{i} \right],
\label{eq:L}
\end{equation}
where $\ns$ is the number of signal events, hence, $(N-\ns)$ the number of
background events in the data sample $D$ of $N$ total events.
The set of source model parameters is denoted as $\ps$. For a point-like source
model, the source model parameters include the source position $\xs$ and the
spectral index $\gamma$ of the source flux.
$S_i(\ps)$ and $B_i$ is the value of the signal and background PDF for the $i$th
data event, respectively.

The signal and background PDFs must incorporate the detector efficiency (yield),
$\mathcal{Y}_i$, which, in general, depends on the celestial direction, the
energy, and the observation time of the data event.

For computational stability reasons the logarithm of the likelihood function of
equation \ref{eq:L} is used in SkyLLH:
\begin{equation}
 \log \mathcal{L}(\ns,\ps~|D) = \sum_{i=1}^{N} \log (...)
\end{equation}

\subsection{Likelihood Ratio Test}

For estimating the significance of an observation, the likelihood ratio
$\Lambda$ with respect to a null hypothesis of no observation, i.e.
equation \ref{eq:L} at $\ns=0$ is tested:
\begin{equation}
 \log \Lambda(\ns,\ps) = \log \frac{\mathcal{L}(\ns,\ps)}{\mathcal{L}(\ns=0)} = \sum_{i=1}^{N} \log \left[ 1 + \frac{\ns}{N}\left( \frac{S_i(\ps)}{B_i} - 1 \right) \right]
\label{eq:logLambda}
\end{equation}
By defining
\begin{equation}
\mathcal{X}_i(\ps) \equiv \frac{1}{N}\left( \mathcal{R}_i(\ps) - 1 \right),
\label{eq:Xi}
\end{equation}
with the signal over background PDF ratio value, $\mathcal{R}_{i}(\ps)$, of the
$i$th event,
\begin{equation}
 \mathcal{R}_i(\ps) \equiv \frac{S_i(\ps)}{B_i},
\end{equation}
this reads as:
\begin{equation}
 \log \Lambda(\ns,\ps) = \sum_{i=1}^{N} \log (1 + \ns\mathcal{X}_i(\ps)).
 \label{eq:logLambdaOfX}
\end{equation}
By defining
\begin{equation}
 \alpha_i(\ns,\ps) \equiv \ns \mathcal{X}_i(\ps)
\end{equation}
the log-likelihood ratio function of the $i$th event can be defined as
\begin{equation}
 \log \Lambda_i(\ns,\ps) \equiv \log(1 + \alpha_i(\ns, \ps)),
 \label{eq:logLambdaiOfalphai}
\end{equation}
and the log-likelihood ratio function for all events can be written as
\begin{equation}
 \log \Lambda(\ns,\ps) = \sum_{i=1}^{N} \log \Lambda_i(\ns,\ps).
 \label{eq:logLambdaOflogLambdai}
\end{equation}

In general, the argument of the $\log$-function, $\Lambda_i$, might become close
to zero, causing a divergence of the log-likelihood ratio function for a particular event.
To circumvent this, a Taylor expansion of the
log-likelihood ratio function of the $i$th event can be performed around a
predefined threshold value $\alpha$.
The event-based log-likelihood ratio function, $\log \Lambda_i$, is then
approximated by a second-order Taylor expansion for events with $\alpha_i \leq \alpha$:
\begin{equation}
 \log\Lambda_i(\ns,\ps) \equiv \log(1 + \alpha) + \frac{\alpha_i(\ns,\ps) - \alpha}{1 + \alpha} - \frac{1}{2} \left(\frac{\alpha_i(\ns,\ps) - \alpha}{1 + \alpha}\right)^2
 \label{eq:logLambdaiTaylor}
\end{equation}
By defining
\begin{equation}
 \tilde{\alpha}_i(\ns,\ps) \equiv \frac{\alpha_i(\ns,\ps) - \alpha}{1 + \alpha},
\end{equation}
the Taylor expanded log-likelihood ratio function reads more compactly:
\begin{equation}
 \log\Lambda_i(\ns,\ps) = \log(1 + \alpha) + \tilde{\alpha}_i(\ns,\ps) - \frac{1}{2} \tilde{\alpha}_i^2(\ns,\ps).
 \label{eq:logLambdaiTaylorOfTildeAlpha}
\end{equation}


\subsection{Test Statistic}
\label{sec:teststatistic}

Assuming Wilks' theorem, a test statistic, TS, for the two-component log-likelihood
ratio test can be formulated using the log-likelihood ratio function, $\log\Lambda$,
at its maximum:
\begin{equation}
 \mathrm{TS} = 2\mathrm{sgn}(\hatns) \log \Lambda(\hatns, \hatps),
 \label{eq:TS}
\end{equation}
where $\log \Lambda(\hatns,\hatps)$ is the maximum of the
log-likelihood ratio function as defined by equation (\ref{eq:logLambda}),
with separation of an over- ($\hatns > 0$) and under-fluctuation ($\hatns < 0$).
In case the assumptions of Wilks' theorem are met within the analysis, the test
statistic value distribution will follow a $\chi^2$-distribution with
a degree-of-freedom equal to the number of fit parameters.

For the case $\hatns=0$, the log-likelihood ratio function is zero
with degenerate source fit parameter values $\hatps$.
When calculating the sensitivity of the analysis, the median test-statistic value
for background-only data is required. Hence, $\hatns$ is often zero in such cases.
Having to deal with a delta-peak of the test-statistic distribution
for $\mathrm{TS}=0$ is cumbersome. In order to resolve the delta-peak, the
log-likelihood ratio function can be approximated with a second-order Taylor
expansion around $\ns=0$, and the apex of that Taylor function defines the value of
log-likelihood ratio function for $\hatns=0$. In this case the test-statistic
function is given by
\begin{equation}
 \mathrm{TS} = -2\frac{\left(\frac{\mathrm{d}\log\Lambda(\ns=0,\ps=\hatps)}{\mathrm{d}\ns}\right)^2}{4\frac{\mathrm{d}^2\log\Lambda(\ns=0,\ps=\hatps)}{\mathrm{d}\ns^2}}.
\end{equation}

\subsection{Optimizations for Spatially Restricted Sources}

For spatially restricted sources, e.g. point-like sources, most of the events in
the data sample will be far away from the hypothesised source, hence, the value of the
signal PDF, $S_i$, will be zero or very close to zero. By selecting only
the signal-contributing $N'$ events from the sample, the log-likelihood ratio
function, $\log \Lambda$, reads
\begin{equation}
 \log \Lambda(\ns,\ps) = \sum_{i=1}^{N'} \log \Lambda_i(\ns,\ps) + (N - N')\log(1 - \frac{\ns}{N}),
 \label{eq:logLambdaOfXOptimized}
\end{equation}
where for $N-N'$ events $\mathcal{R}_i(\ps)$ equals zero and hence
$\alpha_i(\ns,\ps)$ becomes $-\ns/N$, and $\log \Lambda_i$ equals $\log(1 - \ns/N)$
for all such pure background events.

\subsection{Signal PDFs}

The likelihood ratio function as given in equation \eq{eq:logLambda}
incorporates a signal probability, $S_i$, for an individual event $i$, given a
signal hypothesis.
Without loss of generality it can be expressed as a joint probability of a
spatial and an energy p.d.f. at the observation time $t_i$:
\begin{equation}
 S_i(\ps) \equiv S_i(\vec{x}_i, E_i, t_i|\ps) = \mathcal{S}_{\mathrm{S}}(\vec{x}_i|t_i,\ps) \mathcal{E}_{\mathrm{S}}(E_i|\vec{x}_i,t_i,\ps)
\end{equation}
It should be noted that at this stage the time-dependences of the spatial,
$\mathcal{S}_{\mathrm{S}}$, and energy, $\mathcal{E}_{\mathrm{S}}$, p.d.f.s
might still be different. In cases where the hypothesised source flux,
$\Phi_{\mathrm{S}}$, has a spatial and energy independent time profile, i.e.
usually named a ``flare'', the time-dependence factorizes:
\begin{equation}
 S_i(\ps) \equiv S_i(\vec{x}_i, E_i, t_i|\ps) = \mathcal{S}_{\mathrm{S}}(\vec{x}_i|\ps) \mathcal{E}_{\mathrm{S}}(E_i|\vec{x}_i,\ps) \mathcal{T}_{\mathrm{S}}(t_i|\ps).
\end{equation}
In general, the signal model parameters $\ps$ can be divided into spatial,
energy, and time parameters, i.e. $\ps = (\vec{p}_{\mathrm{s,spatial}},
\vec{p}_{\mathrm{s,energy}}, \vec{p}_{\mathrm{s,time}})$:
\begin{equation}
 S_i(\ps) = \mathcal{S}_{\mathrm{S}}(\vec{x}_i|\vec{p}_{\mathrm{s,spatial}}) \mathcal{E}_{\mathrm{S}}(E_i|\vec{x}_i,\vec{p}_{\mathrm{s,energy}}) \mathcal{T}_{\mathrm{S}}(t_i|\vec{p}_{\mathrm{s,time}}),
 \label{eq:Si}
\end{equation}
The spatial component,
$\mathcal{S}_{\mathrm{S}}$, can be identified as the convolution,
$(\Psi \ast \mathrm{PSF})(\alpha,\delta)$\footnote{The 2D convolution on the sky
is defined as
$(f \ast g)(\alpha,\delta) = \int_{0}^{2\pi} \mathrm{d}\alpha' \int_{-\pi}^{\pi} \mathrm{d}\delta' f(\alpha',\delta')g(\alpha-\alpha',\delta-\delta')$.},
of the spatial source extension, $\Psi(\alpha,\delta)$, and the point-spread-function,
$\mathrm{PSF}(\alpha,\delta)$, of the detector.
For a point-like spatial source extension at position
$\xs = (\alpha_{\mathrm{s}},\delta_{\mathrm{s}})$, that is
$\Psi(\alpha,\delta) = \delta(\alpha-\alpha_{\mathrm{s}})\delta(\delta-\delta_{\mathrm{s}})$,
where $\delta(\cdot)$ is the delta-distribution, this convolution collapses to
a single point in the sky. With a 2D Gaussian PSF
$\mathcal{S}_{\mathrm{S}}(\vec{x}_i|\vec{p}_{\mathrm{s,spatial}})$ is given as
\begin{equation}
 \mathcal{S}_{\mathrm{S}}(\vec{x}_i|\vec{p}_{\mathrm{s,spatial}}) \equiv \mathcal{S}_{\mathrm{S}}(r_i,\sigma_i|\xs) = \frac{1}{2\pi\sigma_i^2}\exp\left({-\frac{r_i^2}{2\sigma_i^2}}\right),
\end{equation}
where $r_i$ is the space angle between the source position and the recorded
reconstructed event direction. In equatorial coordinates,
$\vec{x} = (\alpha,\delta)$, the cosine of $r_i$ is given by
\begin{equation}
 \cos(r_i) = \cos(\alpha_{\mathrm{s}} - \alpha_i) \cos(\delta_{\mathrm{s}})\cos({\delta_i}) + \sin(\delta_{\mathrm{s}})\sin(\delta_i).
\end{equation}
The data quantity $\sigma_i$ describes the angular reconstruction uncertainty of
the event, hence the PSF is narrower for well-reconstructed events, and wider
for events which have a large reconstruction uncertainty.

The energy signal PDF can be constructed from Monte-Carlo data using the assumed
source energy spectrum.
When considering a power law as source flux model, the energy source parameters,
$\vec{p}_{\mathrm{s,energy}}$, consists of the spectral index $\gamma$ and possibly
an energy cut-off parameter $E_{\mathrm{cut}}$.

The source time PDF, $\mathcal{T}_{\mathrm{S}}(t|\vec{p}_{\mathrm{s},\mathrm{time}})$,
is the convolution of the source time profile,
$T_{\mathrm{S}}(t|\vec{p}_{\mathrm{s},\mathrm{time}})$, as given in section
\ref{sec:point-like-source-factorized-flux-models}, and the detector live-time function,
\begin{equation}
 T_{\mathrm{live}}(t) = \begin{cases}
                         1 & \forall t \in \text{detector on-time window} \\
                         0 & \text{otherwise}
                        \end{cases},
 \label{eq:Tlive}
\end{equation}
where the final result is normalized to unity.

For the source time emission profiles given in section
\ref{sec:point-like-source-factorized-flux-models} we here state their
normalization factors, $\tilde{T}_{\mathrm{S}}$.

For a steady source profile over the entire observation (live) time,
$T_{\mathrm{obs}}$,
\begin{equation}
 \tilde{T}_{\mathrm{S}} \equiv \frac{1}{T_{\mathrm{obs}}},
\end{equation}
with
\begin{equation}
 T_{\mathrm{obs}} = \int T_{\mathrm{live}}(t) \mathrm{d}t,
 \label{eq:Tobs}
\end{equation}
where $T_{\mathrm{live}}(t)$ is the detector live-time function as defined in
equation (\ref{eq:Tlive}).

For a box profile of length $T_{\mathrm{W}}$ with the box's middle time
position, $T_{0}$, as given by equation (\ref{eq:Ts-box}), the normalization
factor is given by
\begin{equation}
 \tilde{T}_{\mathrm{S}} \equiv
   \begin{cases}
     \frac{1}{T_{\mathrm{W}}} & \forall t \in \left[T_0 - T_{\mathrm{W}}/2; T_0 + T_{\mathrm{W}}/2 \right]\\
     0 & \mathrm{otherwise}
   \end{cases}.
\end{equation}

For a Gaussian shaped time profile centered at $T_0$ with a time width of $\sigma_T$,
as given by equation (\ref{eq:Ts-gauss-non-truncated}),
\begin{equation}
 \tilde{T_{\mathrm{S}}} \equiv \frac{1}{\sqrt{2\pi}\sigma_T}.
\end{equation}


\subsection{Background PDFs}

In analog to the signal PDF, the background PDF can be formulated as
\begin{equation}
 B_i \equiv \mathcal{S}_{\mathrm{B}}(\vec{x}_i|t_i) \mathcal{E}_{\mathrm{B}}(E_i|\vec{x}_i,t_i).
 \label{eq:Bit}
\end{equation}
In cases where the background model has a spatial and energy independent
time-profile, e.g. for a constant background flux, the time-dependence factorizes:
\begin{equation}
 B_i \equiv \mathcal{S}_{\mathrm{B}}(\vec{x}_i) \mathcal{E}_{\mathrm{B}}(E_i|\vec{x}_i) \mathcal{T}_{\mathrm{B}}(t_i).
 \label{eq:Bi}
\end{equation}
All the background PDF components can either be determined from the data itself
or by using Monte-Carlo simulation.

For a background model of a constant background flux the background time profile,
$T_{\mathrm{B}}(t)$, is given by
\begin{equation}
 T_{\mathrm{B}}(t) \equiv \frac{1}{T_{\mathrm{obs}}},
\end{equation}
where $T_{\mathrm{obs}}$ is given by equation (\ref{eq:Tobs}).

In analog to the signal time PDF, the background time PDF, $\mathcal{T}_{\mathrm{B}}(t)$,
is the convolution of the background time profile $T_{\mathrm{B}}(t)$ and the
detector live-time function, $T_{\mathrm{live}}(t)$, where the final result is
again normalized to unity.

On short time-scales the uniform, time-independent, background hypothesis might
not be a reasonable assumption possibly due to a varying event acceptance of the
detector. In such cases the background time profile has to be determined from
simulation or the experimental data itself.


\subsection{Notes on the energy PDFs for Signal \& Background}

In general, the energy PDFs are detector response dependent. That means they
depend on the local direction of the detected events. Hence, the spatial and
energy PDFs cannot be factorized entirely in space and energy.

For IceCube the energy resolution mostly depends on the zenith angle, and hence
on the declination, of the event. Thus, several energy PDFs are created for a
set of (reconstructed) declination bands, both, for signal and background. At
the data evaluation, the signal and background PDFs are selected corresponding
to the declination band the event's declination is part of. Hence, for IceCube,
the signal and background energy PDFs can be formulated as
$\mathcal{E}_{\mathrm{S}}(E|\delta,\vec{p}_{\mathrm{s,energy}})$ and
$\mathcal{E}_{\mathrm{B}}(E|\delta)$, respectively.

A lengthly discussion has been conducted in the past to clarify whether the true or
reconstructed direction of the Monte-Carlo events should be used to generate
the several signal energy PDFs. Since, we mainly use experimental data as
background estimation it as been concluded to use the reconstructed event
direction in order to be consistent in the data evaluation for signal and
background PDFs.

\subsection{Stacking of Sources}

In general a likelihood value can be calculated for a set of $K$ stacked
sources in a weighted fashion. In this case the signal PDF expression of
equation (\ref{eq:Si}) becomes a bit more complicated due to the relative
source weighting. The sources must be weighted according to their signal
detector yield, $\mathcal{Y}_{\mathrm{s}_k}$, and a relative strength weight of
the sources, $W_k$. Hence, the combined signal PDF is given as
\begin{equation}
 \mathcal{S}_i(\ps) \equiv \frac{\sum_{k=1}^{K} W_k \mathcal{Y}_{\mathrm{s}_k}(\psk) \mathcal{S}_{i}(\psk)}{\sum_{k=1}^{K}W_k\mathcal{Y}_{\mathrm{s}_k}(\psk)}.
 \label{eq:SiStacking}
\end{equation}
One should note that this formalism allows for different source properties, e.g.
energy spectra, for the various sources.

With the definitions
\begin{equation}
a_k(\psk) \equiv W_k \mathcal{Y}_{\mathrm{s}_k}(\psk),
\label{eq:SourceWeightCoefficient}
\end{equation}
and
\begin{equation}
A(\ps) \equiv \sum_{k=1}^{K} a_k(\psk),
\label{eq:SumOfSourceWeightCoefficients}
\end{equation}
equation (\ref{eq:SiStacking}) reads
\begin{equation}
\mathcal{S}_i(\ps) \equiv \frac{1}{A(\ps)} \sum_{k=1}^{K} a_k(\psk) \mathcal{S}_{i}(\psk).
\label{eq:SiStackingA}
\end{equation}


\subsection{Gradients of the Log-Likelihood Ratio}

For maximizing the log-likelihood ratio function (equation (\ref{eq:logLambdaOflogLambdai})),
or minimizing the negative of it, the minimizer algorithm requires the
derivatives of the log-likelihood ratio function w.r.t. the fit parameters,
$\ns$ and $\ps$.
Hence, here we provide the expressions of these derivatives
for the optimized log-likelihood ratio function as given by equation
(\ref{eq:logLambdaOfXOptimized}).

The derivative w.r.t. $\ns$ is given by
\begin{equation}
\frac{\mathrm{d} \log \Lambda(\ns,\ps)}{\mathrm{d} \ns} = \sum_{i=1}^{N'} \frac{\mathrm{d} \log \Lambda_i(\ns,\ps)}{\mathrm{d} \ns} - \frac{N - N'}{N - \ns}.
\end{equation}
For the numerical stable case, where $\alpha_i > \alpha$, the derivative of the
log-likelihood ratio function of the $i$th event w.r.t. $\ns$ is given by the
derivative of equation (\ref{eq:logLambdaiOfalphai}) w.r.t. $\ns$:
\begin{equation}
 \frac{\mathrm{d} \log \Lambda_i(\ns,\ps)}{\mathrm{d} \ns} = \frac{\mathcal{X}_i(\ps)}{1+\alpha_i(\ns,\ps)}.
 \label{eq:dlogLambdaidns-for-alphai-gt-alpha}
\end{equation}
For the numerical unstable case, where $\alpha_i \leq \alpha$, this derivative is
given by the derivative of the Taylor expansion of equation (\ref{eq:logLambdaiTaylorOfTildeAlpha})
w.r.t. $\ns$:
\begin{equation}
 \frac{\mathrm{d}\log\Lambda_i(\ns,\ps)}{\mathrm{d} \ns} = \frac{1}{1+\alpha}\left(1 - \tilde{\alpha}_i(\ns,\ps)\right) \mathcal{X}_i(\ps)
 \label{eq:dlogLambdaidns-for-alphai-le-alpha}
\end{equation}

For calculating the test-statistic, \emph{c.f.} section \ref{sec:teststatistic},
the second derivative w.r.t. $\ns$ become in handy for the case $\ns=0$. Hence,
it is provided here as well:
\begin{equation}
 \frac{\mathrm{d}^2\log\Lambda(\ns,\ps)}{\mathrm{d} \ns^2} = \sum_{i=1}^{N'} \frac{\mathrm{d}^2 \log \Lambda_i(\ns,\ps)}{\mathrm{d} \ns^2} - \frac{N-N'}{(N - \ns)^2}
\end{equation}
The second derivative w.r.t. $\ns$ of the log-likelihood ratio function for an
individual event with $\alpha_i > \alpha$ is given by the derivative of equation
(\ref{eq:dlogLambdaidns-for-alphai-gt-alpha}):
\begin{equation}
 \frac{\mathrm{d}^2\log\Lambda_i(\ns,\ps)}{\mathrm{d} \ns^2} = - \left(\frac{\mathcal{X}_i(\ps)}{1+\alpha_i(\ns,\ps)}\right)^2 = - \left( \frac{\mathrm{d} \log \Lambda_i(\ns,\ps)}{\mathrm{d} \ns} \right)^2
 \label{eq:d2logLambdadns2-for-alphai-gt-alpha}
\end{equation}
For the event case $\alpha_i \le \alpha$ this second derivative would be a
constant due to the second-order nature of the chosen Taylor expansion in that
case. At the junction point $\alpha$ the second derivative would not be differentiable.
Hence, equation (\ref{eq:d2logLambdadns2-for-alphai-gt-alpha}) is used as well
in this case, with $\mathrm{d}\log\Lambda_i(\ns,\ps)/\mathrm{d}\ns$ given by
equation (\ref{eq:dlogLambdaidns-for-alphai-le-alpha}). This provides a second
derivative that is differentiable for all $(1 + \alpha_i)$ values, does not
diverge for $(1 + \alpha_i) \rightarrow 0$, and is closer to the second
derivative of $\log\Lambda_i(\ns,\ps)$ for the $\alpha_i > \alpha$ case.

The derivative w.r.t. an individual signal parameter, $p_{\mathrm{s}}$, is given by
\begin{equation}
 \frac{\mathrm{d} \log \Lambda(\ns,\ps)}{\mathrm{d} p_{\mathrm{s}}} = \sum_{i=1}^{N'} \frac{\mathrm{d}\log\Lambda_i(\ns,\ps)}{\mathrm{d}p_{\mathrm{s}}}
\end{equation}
Again, one needs to distinguish between the numerical stable ($\alpha_i > \alpha$)
and unstable ($\alpha_i \leq \alpha$) case.
For the stable case the event-based derivative w.r.t. $p_{\mathrm{s}}$ is given by
\begin{equation}
 \frac{\mathrm{d}\log\Lambda_i(\ns,\ps)}{\mathrm{d}p_{\mathrm{s}}} = \frac{\ns}{1+\alpha_i(\ns,\ps)} \frac{\mathrm{d}\mathcal{X}_i(\ps)}{\mathrm{d}p_{\mathrm{s}}}.
\end{equation}
For the numerical unstable case this derivative is
given by the derivative of the Taylor expansion of equation (\ref{eq:logLambdaiTaylorOfTildeAlpha})
w.r.t. $p_{\mathrm{s}}$:
\begin{equation}
 \frac{\mathrm{d}\log\Lambda_i(\ns,\ps)}{\mathrm{d}p_{\mathrm{s}}} = \frac{\ns}{1+\alpha}\left(1 - \tilde{\alpha}_i(\ns,\ps)\right) \frac{\mathrm{d}\mathcal{X}_i(\ps)}{\mathrm{d}p_{\mathrm{s}}}.
\end{equation}

The derivative of $\mathcal{X}_i$ can be calculated using
equation \ref{eq:Xi} and the expressions for the signal and background PDFs as given
in equation \ref{eq:Si} and \ref{eq:Bi}, respectively. Depending on the type of
fit parameter, i.e. spatial, energy, or time, the derivative of the PDF ratio,
$\mathcal{R}_i(\ps) = \mathcal{S}_i(\ps) / \mathcal{B}_i$, simplifies to the
derivative of the respective type of PDF ratio:
\begin{equation}
 \frac{\mathrm{d} \mathcal{X}_i(\ps)}{\mathrm{d} p_{\mathrm{s}}} = \frac{1}{N}\frac{\mathrm{d} \mathcal{R}_i(\ps)}{\mathrm{d} p_{\mathrm{s}}},
\end{equation}
with
\begin{equation}
 \mathcal{R}_i(\ps) = \mathcal{R}_{S,i}(\vec{p}_{s,\mathrm{spatial}}) \mathcal{R}_{\mathcal{E},i}(\vec{p}_{s,\mathrm{energy}}) \mathcal{R}_{\mathcal{T},i}(\vec{p}_{s,\mathrm{time}}),
 \label{eq:Ri}
\end{equation}
and
\begin{equation}
 \frac{\mathrm{d} \mathcal{R}_i(\ps)}{\mathrm{d} p_{\mathrm{s,spatial}}} = \frac{\mathrm{d} \mathcal{R}_{S,i}(\vec{p}_{s,\mathrm{spatial}})}{\mathrm{d} p_{\mathrm{s,spatial}}} \mathcal{R}_{\mathcal{E},i}(\vec{p}_{s,\mathrm{energy}}) \mathcal{R}_{\mathcal{T},i}(\vec{p}_{s,\mathrm{time}}),
\end{equation}
\begin{equation}
 \frac{\mathrm{d} \mathcal{R}_i(\ps)}{\mathrm{d} p_{\mathrm{s,energy}}} = \mathcal{R}_{S,i}(\vec{p}_{s,\mathrm{spatial}}) \frac{\mathrm{d} \mathcal{R}_{\mathcal{E},i}(\vec{p}_{s,\mathrm{energy}})}{\mathrm{d} p_{\mathrm{s,energy}}} \mathcal{R}_{\mathcal{T},i}(\vec{p}_{s,\mathrm{time}}),
\end{equation}
\begin{equation}
 \frac{\mathrm{d} \mathcal{R}_i(\ps)}{\mathrm{d} p_{\mathrm{s,time}}} = \mathcal{R}_{S,i}(\vec{p}_{s,\mathrm{spatial}}) \mathcal{R}_{\mathcal{E},i}(\vec{p}_{s,\mathrm{energy}}) \frac{\mathrm{d} \mathcal{R}_{\mathcal{T},i}(\vec{p}_{s,\mathrm{time}})}{\mathrm{d} p_{\mathrm{s,time}}}.
\end{equation}

For stacked sources the expression for $\mathcal{R}_i(\ps)$ in equation (\ref{eq:Ri})
becomes slightly more complicated due to the source strength weighting.
With equation (\ref{eq:SiStacking}) and the definitions
(\ref{eq:SourceWeightCoefficient}), and
(\ref{eq:SumOfSourceWeightCoefficients}), it is given by
\begin{equation}
\mathcal{R}_i(\ps) = \frac{\mathcal{S}_i(\ps)}{\mathcal{B}_i} = \frac{1}{A(\ps)} \sum_{k=1}^{K} a_k(\psk) \frac{\mathcal{S}_{i}(\psk)}{\mathcal{B}_{i}}.
\label{eq:RiStacking}
\end{equation}
The signal over background ratio $\mathcal{S}_{i}(\psk) / \mathcal{B}_{i} \equiv \mathcal{R}_{i,k}(\psk)$
for the source $k$ is then given by equation (\ref{eq:Ri}).

Using the same set of source fit parameters $\ps$ for all sources, the
derivative of $\mathcal{R}_i(\ps)$ for
all stacked sources w.r.t. the single source fit parameter,
$p_{\mathrm{s}}$, is then given by
\begin{equation}
 \frac{\mathrm{d} \mathcal{R}_{i}(\ps)}{\mathrm{d} p_{\mathrm{s}}} = - \frac{1}{A^2} \frac{\mathrm{d} A}{\mathrm{d} p_{\mathrm{s}}} \sum_{k=1}^{K} a_{k} \mathcal{R}_{i,k}(\psk) + \frac{1}{A}\sum_{k=1}^{K} \left( \frac{\mathrm{d} a_{k}}{\mathrm{d} p_{\mathrm{s}}}\mathcal{R}_{i,k}(\psk) + a_{k}\frac{\mathrm{d} \mathcal{R}_{i,k}(\psk)}{\mathrm{d} p_{\mathrm{s}}} \right).
\end{equation}
Using $\mathcal{R}_i(\ps)$ from equation (\ref{eq:RiStacking}) it simplifies to
\begin{equation}
 \frac{\mathrm{d} \mathcal{R}_{i}(\ps)}{\mathrm{d} p_{\mathrm{s}}} = \frac{1}{A(\ps)}\left[ -\mathcal{R}_i(\ps)\frac{\mathrm{d} A}{\mathrm{d} p_{\mathrm{s}}} + \sum_{k=1}^{K} \left( \frac{\mathrm{d} a_{k}}{\mathrm{d} p_{\mathrm{s}}}\mathcal{R}_{i,k}(\psk) + a_{k}\frac{\mathrm{d} \mathcal{R}_{i,k}(\psk)}{\mathrm{d} p_{\mathrm{s}}} \right) \right],
 \label{eq:gradRi}
\end{equation}
with the derivative of equation (\ref{eq:SumOfSourceWeightCoefficients}) given by
\begin{equation}
 \frac{\mathrm{d} A}{\mathrm{d} p_{\mathrm{s}}} = \sum_{k=1}^{K} \frac{\mathrm{d} a_k}{\mathrm{d} p_{\mathrm{s}}},
\end{equation}
with
\begin{equation}
\frac{\mathrm{d} a_k}{\mathrm{d} p_{\mathrm{s}}} = W_k \frac{\mathrm{d}\mathcal{Y}_{\mathrm{s}_{k}}(\psk)}{\mathrm{d}{p_s}_k},
\end{equation}
where the local source parameter $p_{\mathrm{s}_{k}}$ maps to the correct global
parameter $p_{\mathrm{s}}$.

In case one would fit each source individually with its own set of source fit
parameters, $\psk$, $\ps$ would be a set of $K$ sets
of source fit parameters $\psk$, and a derivative for each
global fit parameter $\ps$ would have to be calculated.
The expression for such a derivative would be equivalent to equation
(\ref{eq:gradRi}), but only the summand for the source, which depend on the
particular global fit parameter, would contribute.


\subsection{Multiple Datasets}

With SkyLLH a set of $J$ different data samples (datasets), $\mathrm{D}_j$, can be
analyzed at once. Each data sample has its own detector signal yield,
$\mathcal{Y}_{\mathrm{s}_j}$.

The composite likelihood function is the product of the individual dataset
likelihood functions:
\begin{equation}
 \log \Lambda = \sum_{j=1}^{J} \log \Lambda_j.
 \label{eq:logLambdaComposite}
\end{equation}

The total number of signal events $\ns$ needs to get split-up into
$n_{\mathrm{s}_j}$ for the individual datasets. The distribution of $\ns$
along the different datasets is based on the detector signal yield,
$\mathcal{Y}_{\mathrm{s}_j}$, of each dataset. For a single source it is given by:
\begin{equation}
 n_{\mathrm{s}_j}(\ns,\ps) = \ns \frac{\mathcal{Y}_{\mathrm{s}_j}(\xs,\ps)}{\sum_{j'=1}^{J} \mathcal{Y}_{\mathrm{s}_{j'}}(\xs,\ps)},
 \label{eq:nsjy}
\end{equation}
where $\xs$ and $\ps$ denote the source position and flux fit parameters, e.g.
the spectral index $\gamma$, respectively. The detector signal yield can be
calculated via the detector effective area and the source flux (\emph{c.f.} section
\ref{sec:detsigyield}).

By defining the dataset weight factor
\begin{equation}
 f_j(\ps) \equiv \frac{\mathcal{Y}_{\mathrm{s}_j}(\xs,\ps)}{\sum_{j'=1}^{J} \mathcal{Y}_{\mathrm{s}_{j'}}(\xs,\ps)}
 \label{eq:dataset-weight-factor-single-source}
\end{equation}
with the property
\begin{equation}
 \sum_{j=1}^{J} f_j = 1
\end{equation}
equation \ref{eq:nsjy} reads
\begin{equation}
 n_{\mathrm{s}_j}(\ns,\ps) = \ns f_{j}(\ps)
 \label{eq:ns-sample-weight-factor}
\end{equation}

Using the dataset weight factor $f_{j}(\ps)$ the likelihood ratio of
equation \eq{eq:logLambdaComposite} with equation \eq{eq:logLambdaOfX} can now
be written as
\begin{equation}
 \log \Lambda(\ns,\ps) = \sum_{j=1}^{J} \sum_{i=1}^{N} \log (1 + \ns f_{j}(\ps)\mathcal{X}_i(\ps)).
\end{equation}
From a reusability-of-software point of view it is advisable to be able to use
the mathematical form of $\log \Lambda$ for the single dataset to calculate the
combined $\log \Lambda$ value of the multiple dataset. This can be achieved by
using the substitution of $\ns$ as given by equation (\ref{eq:ns-sample-weight-factor}).
Hence,
\begin{equation}
 \log \Lambda(\ns,\ps) = \sum_{j=1}^{J} \log \Lambda_j(n_{\mathrm{s}_j}(\ns,\ps),\ps).
 \label{eq:logLambdaOfLogLambdaj}
\end{equation}

For multiple point sources, i.e. a stacking of $K$ point sources with positions
$\xsk$, the dataset weight factor of each single source needs
to be taken into account via Bayes' theorem. Thus, $f_{j}(\ps)$ can be written
as the sum of the products of the dataset weight factor $f_{j}(\psk)$ for
source $k$, as given by equation (\ref{eq:dataset-weight-factor-single-source}),
and the relative strength, $f_{k}(\psk)$, of the $k$th source in all datasets
compared to all the other sources in all datasets:
\begin{equation}
 f_{j}(\ps) = \sum_{k=1}^{K} f_{k}(\psk) f_{j}(\psk).
\end{equation}
The relative strength of source $k$ can be written as
\begin{equation}
 f_{k}(\psk) = \frac{\sum_{j=1}^{J} \mathcal{Y}_{\mathrm{s}_{j,k}}(\xsk,\psk)}{\sum_{\kappa=1}^{K} \sum_{j=1}^{J} \mathcal{Y}_{\mathrm{s}_{j,\kappa}}(\vec{x}_{\mathrm{s}_\kappa},\vec{p}_{\mathrm{s}_\kappa}) }
 \label{eq:fk}
\end{equation}
By combining equation \ref{eq:dataset-weight-factor-single-source} with $\xs \equiv \xsk$
and $\ps \equiv \psk$, and equation \ref{eq:fk}, the expression for
$f_{j}(\ps)$ for multiple sources is given by:
\begin{equation}
 f_{j}(\ps) = \sum_{k=1}^{K}
    \frac{\left(\sum_{j'=1}^{J} \mathcal{Y}_{\mathrm{s}_{j',k}}(\xsk,\psk)\right) \mathcal{Y}_{\mathrm{s}_{j,k}}(\xsk,\psk)}
         {\left(\sum_{\kappa=1}^{K} \sum_{j'=1}^{J} \mathcal{Y}_{\mathrm{s}_{j',\kappa}}(\vec{x}_{\mathrm{s}_\kappa},\vec{p}_{\mathrm{s}_\kappa})\right) \left( \sum_{j'=1}^{J} \mathcal{Y}_{\mathrm{s}_{j',k}}(\xsk,\psk) \right)}
\end{equation}
The sum over the datasets of the detector signal yield for source $k$ cancels
out leaving the simplified equation
\begin{equation}
 f_{j}(\ps) = \frac{\sum_{k=1}^{K} \mathcal{Y}_{\mathrm{s}_{j,k}}(\xsk,\psk)}
                   {\sum_{k=1}^{K} \sum_{j'=1}^{J} \mathcal{Y}_{\mathrm{s}_{j',k}}(\xsk,\psk)}.
 \label{eq:dataset-weight-factor-multiple-sources}
\end{equation}

\subsection{Gradients of the Multi-Dataset Log-Likelihood Ratio}

By using equation (\ref{eq:logLambdaOfLogLambdaj}) for the combined log-likelihood
ratio, its derivative w.r.t. $\ns$ is given by
\begin{equation}
 \frac{\mathrm{d}\log\Lambda(\ns,\ps)}{\mathrm{d}\ns} = \sum_{j=1}^{J} \frac{\mathrm{d}\log\Lambda_j(n_{\mathrm{s}_j},\ps)}{\mathrm{d}n_{\mathrm{s}_j}} \frac{\mathrm{d} n_{\mathrm{s}_j}}{\mathrm{d} \ns},
\end{equation}
with
\begin{equation}
\frac{\mathrm{d} n_{\mathrm{s}_j}}{\mathrm{d} \ns} = f_j(\ps).
\end{equation}
Its second derivative w.r.t. $\ns$ is given by
\begin{align}
 \frac{\mathrm{d}^2\log\Lambda(\ns,\ps)}{\mathrm{d}\ns^2} &= \sum_{j=1}^{J} \frac{\mathrm{d}}{\mathrm{d}\ns}\left( \frac{\mathrm{d}\log\Lambda_j(n_{\mathrm{s}_j},\ps)}{\mathrm{d}n_{\mathrm{s}_j}} \right) \frac{\mathrm{d} n_{\mathrm{s}_j}}{\mathrm{d} \ns}\\
							  &= \sum_{j=1}^{J} \frac{\mathrm{d}^2\log\Lambda_j(n_{\mathrm{s}_j},\ps)}{\mathrm{d}n_{\mathrm{s}_j}^2} \left( \frac{\mathrm{d} n_{\mathrm{s}_j}}{\mathrm{d} \ns} \right)^2.
\end{align}

The derivative w.r.t. a single source fit parameter, $p_{\mathrm{s}}$, consists
of the partial derivatives of $\log \Lambda_j$ w.r.t. $n_{\mathrm{s}_j}$ and
$p_{\mathrm{s}}$:
\begin{equation}
 \frac{\mathrm{d} \log \Lambda(\ns,\ps)}{\mathrm{d} p_{\mathrm{s}}} = \sum_{j=1}^{J} \left( \frac{\partial \log \Lambda_j(\nsj,\ps)}{\partial \nsj} \frac{\mathrm{d} \nsj}{\mathrm{d} p_{\mathrm{s}}} + \frac{\partial \log \Lambda_j(\nsj,\ps)}{\partial p_{\mathrm{s}}} \right),
\end{equation}
with
\begin{equation}
 \frac{\mathrm{d} \nsj}{\mathrm{d} p_{\mathrm{s}}} = \ns \frac{\mathrm{d} f_j(\ps)}{\mathrm{d} p_{\mathrm{s}}}.
\end{equation}
In case of a single source, the expression for the derivative of the dataset
weight factor, where $f_j(\ps)$ is given by equation (\ref{eq:dataset-weight-factor-single-source}),
reads via the quotient rule of differentiation:
\begin{equation}
\frac{\mathrm{d}f_j(\ps)}{\mathrm{d}p_{\mathrm{s}}} = \frac{\frac{\mathrm{d}\mathcal{Y}_{\mathrm{s}_j}(\xs,\ps)}{\mathrm{d}p_{\mathrm{s}}} \sum_{j'=1}^{J}\mathcal{Y}_{\mathrm{s}_{j'}}(\xs,\ps) - \mathcal{Y}_{\mathrm{s}_j}(\xs,\ps) \sum_{j'=1}^{J} \frac{\mathrm{d}\mathcal{Y}_{\mathrm{s}_{j'}}(\xs,\ps)}{\mathrm{d}p_{\mathrm{s}}} }{\left( \sum_{j'=1}^{J} \mathcal{Y}_{\mathrm{s}_{j'}}(\xs,\ps) \right)^2}.
\end{equation}
In case of multiple sources (stacking), the expression for the derivative of the
dataset weight factor, where $f_j(\ps)$ is given by equation
(\ref{eq:dataset-weight-factor-multiple-sources}) reads via the quotient rule of
differentiation:
\begin{equation}
 \frac{\mathrm{d} f_j(\ps)}{\mathrm{d}p_{\mathrm{s}}} =
    \frac{\left(\sum_{k=1}^{K} \frac{\mathrm{d}\mathcal{Y}_{\mathrm{s}_{j,k}}}{\mathrm{d}p_{\mathrm{s}}}\right) \left(\sum_{k=1}^{K}\sum_{j'=1}^{J}\mathcal{Y}_{\mathrm{s}_{j',k}}\right) - \left(\sum_{k=1}^{K}\mathcal{Y}_{\mathrm{s}_{j,k}}\right)\left(\sum_{k=1}^{K}\sum_{j'=1}^{J}\frac{\mathrm{d}\mathcal{Y}_{\mathrm{s}_{j',k}}}{\mathrm{d}p_{\mathrm{s}}}\right)}
         {\left(\sum_{k=1}^{K}\sum_{j'=1}^{J}\mathcal{Y}_{\mathrm{s}_{j',k}} \right)^2}
\end{equation}

\section{Detector Signal Yield}
\label{sec:detsigyield}

The detector signal yield, $\mathcal{Y}_{\mathrm{s}_{j,k}}(\xsk,\psk)$,
is the mean number of expected signal events in the detector from a given
source $k$ in a given data sample $j$. For a differential source flux,
$\mathrm{d}\Phi_{\mathrm{s}}/(\mathrm{d}A\mathrm{d}\Omega\mathrm{d}E\mathrm{d}t)$,
it is the integral of the product of the detector effective area and this
differential flux over the solid-angle, energy, and time of the source:
\begin{equation}
 \mathcal{Y}_{\mathrm{s}_{j,k}}(\xsk,\psk) \equiv \int_{\Omega_{\mathrm{s}_k}} \mathrm{d}\Omega \int_0^\infty \mathrm{d}E \int_{t_{\mathrm{start}_j}}^{t_{\mathrm{end}_j}}\mathrm{d}t A_{\mathrm{eff}_j}(E,t|\xsk) \frac{\mathrm{d}\Phi_{\mathrm{s}}(E,t|\psk)}{\mathrm{d}A\mathrm{d}\Omega\mathrm{d}E\mathrm{d}t}
\label{eq:Ysj}
\end{equation}
In the most-general case, the source position $\xs$ consists of three
quantities: right-ascension, declination, and observation time, i.e.
$\xs = (\alpha_{\mathrm{s}},\delta_{\mathrm{s}},t_{\mathrm{obs}})$.

The time-dependent effective area $A_{\mathrm{eff}_j}(E,t|\xsk)$ must account
for the detector off-time intervals within the data sample $j$. In cases, where
the effective area is constant within a data sample, it can be written as
\begin{equation}
 A_{\mathrm{eff}_j}(E,t|\xsk) = A_{\mathrm{eff}_j}(E|\xsk) T_{\mathrm{live}}(t)
\end{equation}
with $T_{\mathrm{live}}(t)$ is the detector live-time function as given by
equation (\ref{eq:Tlive}).

\subsection{Effective Area}

In SkyLLH the effective area $A_{\mathrm{eff},j}$ of a data sample $j$ is not
calculated separately in order to avoid binning effects. However, the effective
area can be calculated using the Monte-Carlo weights, \code{mcweight}\footnote{In IceCube
known as ``OneWeight'', but which already includes the number of used MC files.},
of the simulation events.
The Monte-Carlo weights have the unit GeV~cm$^2$~sr.
Using the Monte-Carlo weight, $w_{m,j}$, of the $m$th event of data sample $j$,
that corresponds to a signal event, i.e. an event that has similar properties as
a signal event (\emph{e.g.} same true direction), the effective area is given by
the sum of the weights of those events, divided by the
solid angle and the energy range $\Delta E$ of the summed selected events:
\begin{equation}
 A_{\mathrm{eff}_j}(E) = \frac{\sum_{m=1}^{M} w_{m,j}}{\Omega \Delta E}
\end{equation}


\subsection{The DetSigYield Class}

\class{DetSigYield} provides an abstract base class for a detector signal yield
class to compute the integral given in equation \eq{eq:Ysj}. The detector signal
yield depends on the flux model and its source parameters, which might
change during the likelihood maximization process. It is also dependent on the
detector effective area, hence is detector dependent. Thus,
\class{DetSigYield} must be provided with a detector signal
yield implementation method derived from the \class{DetSigYieldImplMethod}
class.

Detector signal yield values can be retrieved via the call operator
\code{\_\_call\_\_(src, src\_flux\_params)}, which takes the celestial source
position(s), and the additional source flux parameters as arguments.

\subsubsection{The DetSigYieldImplMethod Class}

\class{DetSigYieldImplMethod} is an abstract base class and defines the interface
between the detector signal yield implementation method and the
\class{DetSigYield} class.

% List of detector signal yield implementation methods.
Table \ref{tbl:I3DetSigYieldImplMethod} lists all available IceCube specific
detector signal yield implementation methods and their description.
\begin{table}
\caption{IceCube specific detector signal yield implementation methods.}
\label{tbl:I3DetSigYieldImplMethod}

\begin{tabular}{p{.95\textwidth}}
\hline
Name of Class \& Description \\
\hline
\textbf{FixedFluxPointLikeSourceI3DetSigYieldImplMethod}

IceCube detector signal yield
    implementation method for a fixed flux model and a point-like source.
    The flux model might contain flux parameters, but these are not fit in the
    likelihood maximization process.
    This implementation assumes that the detector effective area depends solely
    on the declination of the source. This method creates
    a spline function of given order for the logarithmic values of the
    $\sin(\delta)$-dependent detector signal yield.

    The constructor of this implementation method requires a $\sin(\delta)$
    binning definition for the Monte-Carlo events and the order of the spline
    function.

    This implementation method create a detector signal yield instance of class
    \class{FixedFluxPointLikeSourceI3DetSigYield}.
\\
\hline
\textbf{PowerLawFluxPointLikeSourceI3DetSigYieldImplMethod}

IceCube detector signal
    yield implementation method for a
    power law flux model, implemented by the \class{PowerLawFlux} class, an a
    point-like source.
    This method creates a 2D spline function of given orders for the logarithmic
    values of the $\sin(\delta)$-dependent detector signal yield for a
    range of $\gamma$ values. This implementation method supports
    multi-processing.
\end{tabular}
\end{table}

\section{The Concept of Source Hypothesis Groups}

The analyses in SkyLLH rely heavily on the calculation of detector signal
efficiencies. As seen in section \ref{sec:detsigyield}, the detector signal
efficiency depends on the source hypothesis (spatial model and flux model)
and the detector response (dataset). Hence, the analyses require detector signal
efficiency instances for each combination of source and dataset.
However, the sources might be of the same kind, i.e. having the same spatial
model and the same flux model. For such sources detector signal efficiency
instances are needed only for each dataset. Thus, we define a group
of sources with the same spatial model and flux model as a \emph{source hypothesis group},
$G_{\mathrm{s}}$.

A source hypothesis group has a list of spatial source models, e.g. point-like source
locations in case of point-like sources, a flux model, and a detection signal
efficiency implementation method assigned.

SkyLLH provides the \class{SourceHypoGroupManager} class to define the groups of
source hypotheses.

\section{Implemented Log-Likelihood Models}
This section describes the implemented log-likelihood models. \cite{TimeDepPSSearchMethods2010}

% \subsection{Time Dependent Point-Source Flare}
%
% The \class{TimeDepPSFlareLHModel} class provides the likelihood model for searching for a point source with unknown time-dependence.
% The search is based on the formulism described in \cite{TimeDepPSSearchMethods2010}.
%
% The model utilizes a two-component likelihood function with signal and background events.

\bibliographystyle{unsrt}
\bibliography{biblio}

\end{document}
